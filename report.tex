% Options for packages loaded elsewhere
\PassOptionsToPackage{unicode}{hyperref}
\PassOptionsToPackage{hyphens}{url}
%
\documentclass[
]{article}
\usepackage{lmodern}
\usepackage{amssymb,amsmath}
\usepackage{ifxetex,ifluatex}
\ifnum 0\ifxetex 1\fi\ifluatex 1\fi=0 % if pdftex
  \usepackage[T1]{fontenc}
  \usepackage[utf8]{inputenc}
  \usepackage{textcomp} % provide euro and other symbols
\else % if luatex or xetex
  \usepackage{unicode-math}
  \defaultfontfeatures{Scale=MatchLowercase}
  \defaultfontfeatures[\rmfamily]{Ligatures=TeX,Scale=1}
\fi
% Use upquote if available, for straight quotes in verbatim environments
\IfFileExists{upquote.sty}{\usepackage{upquote}}{}
\IfFileExists{microtype.sty}{% use microtype if available
  \usepackage[]{microtype}
  \UseMicrotypeSet[protrusion]{basicmath} % disable protrusion for tt fonts
}{}
\makeatletter
\@ifundefined{KOMAClassName}{% if non-KOMA class
  \IfFileExists{parskip.sty}{%
    \usepackage{parskip}
  }{% else
    \setlength{\parindent}{0pt}
    \setlength{\parskip}{6pt plus 2pt minus 1pt}}
}{% if KOMA class
  \KOMAoptions{parskip=half}}
\makeatother
\usepackage{xcolor}
\IfFileExists{xurl.sty}{\usepackage{xurl}}{} % add URL line breaks if available
\IfFileExists{bookmark.sty}{\usepackage{bookmark}}{\usepackage{hyperref}}
\hypersetup{
  hidelinks,
  pdfcreator={LaTeX via pandoc}}
\urlstyle{same} % disable monospaced font for URLs
\usepackage{color}
\usepackage{fancyvrb}
\newcommand{\VerbBar}{|}
\newcommand{\VERB}{\Verb[commandchars=\\\{\}]}
\DefineVerbatimEnvironment{Highlighting}{Verbatim}{commandchars=\\\{\}}
% Add ',fontsize=\small' for more characters per line
\newenvironment{Shaded}{}{}
\newcommand{\AlertTok}[1]{\textcolor[rgb]{1.00,0.00,0.00}{\textbf{#1}}}
\newcommand{\AnnotationTok}[1]{\textcolor[rgb]{0.38,0.63,0.69}{\textbf{\textit{#1}}}}
\newcommand{\AttributeTok}[1]{\textcolor[rgb]{0.49,0.56,0.16}{#1}}
\newcommand{\BaseNTok}[1]{\textcolor[rgb]{0.25,0.63,0.44}{#1}}
\newcommand{\BuiltInTok}[1]{#1}
\newcommand{\CharTok}[1]{\textcolor[rgb]{0.25,0.44,0.63}{#1}}
\newcommand{\CommentTok}[1]{\textcolor[rgb]{0.38,0.63,0.69}{\textit{#1}}}
\newcommand{\CommentVarTok}[1]{\textcolor[rgb]{0.38,0.63,0.69}{\textbf{\textit{#1}}}}
\newcommand{\ConstantTok}[1]{\textcolor[rgb]{0.53,0.00,0.00}{#1}}
\newcommand{\ControlFlowTok}[1]{\textcolor[rgb]{0.00,0.44,0.13}{\textbf{#1}}}
\newcommand{\DataTypeTok}[1]{\textcolor[rgb]{0.56,0.13,0.00}{#1}}
\newcommand{\DecValTok}[1]{\textcolor[rgb]{0.25,0.63,0.44}{#1}}
\newcommand{\DocumentationTok}[1]{\textcolor[rgb]{0.73,0.13,0.13}{\textit{#1}}}
\newcommand{\ErrorTok}[1]{\textcolor[rgb]{1.00,0.00,0.00}{\textbf{#1}}}
\newcommand{\ExtensionTok}[1]{#1}
\newcommand{\FloatTok}[1]{\textcolor[rgb]{0.25,0.63,0.44}{#1}}
\newcommand{\FunctionTok}[1]{\textcolor[rgb]{0.02,0.16,0.49}{#1}}
\newcommand{\ImportTok}[1]{#1}
\newcommand{\InformationTok}[1]{\textcolor[rgb]{0.38,0.63,0.69}{\textbf{\textit{#1}}}}
\newcommand{\KeywordTok}[1]{\textcolor[rgb]{0.00,0.44,0.13}{\textbf{#1}}}
\newcommand{\NormalTok}[1]{#1}
\newcommand{\OperatorTok}[1]{\textcolor[rgb]{0.40,0.40,0.40}{#1}}
\newcommand{\OtherTok}[1]{\textcolor[rgb]{0.00,0.44,0.13}{#1}}
\newcommand{\PreprocessorTok}[1]{\textcolor[rgb]{0.74,0.48,0.00}{#1}}
\newcommand{\RegionMarkerTok}[1]{#1}
\newcommand{\SpecialCharTok}[1]{\textcolor[rgb]{0.25,0.44,0.63}{#1}}
\newcommand{\SpecialStringTok}[1]{\textcolor[rgb]{0.73,0.40,0.53}{#1}}
\newcommand{\StringTok}[1]{\textcolor[rgb]{0.25,0.44,0.63}{#1}}
\newcommand{\VariableTok}[1]{\textcolor[rgb]{0.10,0.09,0.49}{#1}}
\newcommand{\VerbatimStringTok}[1]{\textcolor[rgb]{0.25,0.44,0.63}{#1}}
\newcommand{\WarningTok}[1]{\textcolor[rgb]{0.38,0.63,0.69}{\textbf{\textit{#1}}}}
\setlength{\emergencystretch}{3em} % prevent overfull lines
\providecommand{\tightlist}{%
  \setlength{\itemsep}{0pt}\setlength{\parskip}{0pt}}
\setcounter{secnumdepth}{-\maxdimen} % remove section numbering

\date{}

\begin{document}

Project Report

\hypertarget{table-of-contents}{%
\subsection{Table of Contents}\label{table-of-contents}}

\begin{itemize}
\tightlist
\item
  \protect\hyperlink{table-of-contents}{Table of Contents}
\item
  \protect\hyperlink{maincpp}{Main.cpp}
\item
  \protect\hyperlink{city-implementation}{City Implementation}

  \begin{itemize}
  \tightlist
  \item
    \protect\hyperlink{city-data-structure}{City Data Structure}
  \item
    \protect\hyperlink{city-interface}{City Interface}
  \end{itemize}
\item
  \protect\hyperlink{min-heap}{Min Heap}

  \begin{itemize}
  \tightlist
  \item
    \protect\hyperlink{min-heap-data-structure}{Min Heap Data Structure}
  \item
    \protect\hyperlink{min-heap-interface}{Min Heap Interface}
  \end{itemize}
\item
  \protect\hyperlink{red-black-tree}{Red Black Tree}

  \begin{itemize}
  \tightlist
  \item
    \protect\hyperlink{red-black-tree-node-structure}{Red black Tree
    Node Structure}
  \item
    \protect\hyperlink{red-black-tree-interface}{Red Black Tree
    Interface}
  \end{itemize}
\end{itemize}

\hypertarget{main.cpp}{%
\subsection{Main.cpp}\label{main.cpp}}

\hypertarget{city-implementation}{%
\subsection{City Implementation}\label{city-implementation}}

\hypertarget{city-data-structure}{%
\subsubsection{City Data Structure}\label{city-data-structure}}

\begin{Shaded}
\begin{Highlighting}[]
\DataTypeTok{int}\NormalTok{ global_time;}
\NormalTok{RBTree *rbTree;}
\NormalTok{Heap *heap;}
\end{Highlighting}
\end{Shaded}

City contains \texttt{global\_time} that keeps track of the global
counter. There's a pointer reference to Red black tree( \texttt{rbTree})
and a min heap(\texttt{heap}) to interact with respective data structure

\hypertarget{city-interface}{%
\subsubsection{City Interface}\label{city-interface}}

\begin{Shaded}
\begin{Highlighting}[]
\DataTypeTok{void}\NormalTok{ insert(}\DataTypeTok{int}\NormalTok{ building_num, }\DataTypeTok{int}\NormalTok{ executed_time, }\DataTypeTok{int}\NormalTok{ total_time);}
\end{Highlighting}
\end{Shaded}

Accepts \texttt{building\_num}, \texttt{executed\_time} \&
\texttt{total\_time} and inserts it into min heap and red black tree.
Acts as an interface between the input and data structures(min heap and
red black).

\begin{Shaded}
\begin{Highlighting}[]
\DataTypeTok{void}\NormalTok{ printBuilding(}\DataTypeTok{int}\NormalTok{ buildingNum);}
\end{Highlighting}
\end{Shaded}

Accepts \texttt{buildingNum} to be printed. Looks up for the given
building number and prints it.

\begin{Shaded}
\begin{Highlighting}[]
\DataTypeTok{void}\NormalTok{ printBuilding(}\DataTypeTok{int}\NormalTok{ from, }\DataTypeTok{int}\NormalTok{ to);}
\end{Highlighting}
\end{Shaded}

Accepts a range from \texttt{from} to \texttt{to} of building number to
be printed. Looks up for the given range and prints it.

\begin{Shaded}
\begin{Highlighting}[]
\DataTypeTok{void}\NormalTok{ updateGlobalTimer(}\DataTypeTok{int}\NormalTok{ time);}
\end{Highlighting}
\end{Shaded}

Setter to modify the \texttt{global\_time} by given \texttt{time}

\begin{Shaded}
\begin{Highlighting}[]
\DataTypeTok{bool}\NormalTok{ hasBuildingToWorkOn();}
\end{Highlighting}
\end{Shaded}

Checks if there any building that could be worked on i.e Checks for the
heap size. Returns true if there are more than more element inside the
min heap data structure otherwise false.

\begin{Shaded}
\begin{Highlighting}[]
\DataTypeTok{int}\NormalTok{ getGlobalTime();}
\end{Highlighting}
\end{Shaded}

Getter for \texttt{global\_time}.

\begin{Shaded}
\begin{Highlighting}[]
\DataTypeTok{int}\NormalTok{ workOnBuilding(}\DataTypeTok{int}\NormalTok{ time);}
\end{Highlighting}
\end{Shaded}

Selects the smallest element from the min heap and either works till the
building is completed or for given \texttt{time} period.

\hypertarget{min-heap}{%
\subsection{Min Heap}\label{min-heap}}

\hypertarget{min-heap-data-structure}{%
\subsubsection{Min Heap Data Structure}\label{min-heap-data-structure}}

\begin{Shaded}
\begin{Highlighting}[]
\KeywordTok{struct}\NormalTok{ Building\{}
    \DataTypeTok{int}\NormalTok{ building_num;}
    \DataTypeTok{int}\NormalTok{ executed_time;}
    \DataTypeTok{int}\NormalTok{ total_time;}
    \KeywordTok{struct}\NormalTok{ Node* twin;}
\NormalTok{\};}
\end{Highlighting}
\end{Shaded}

Each element in min heap is an instance of \texttt{Building}.
\texttt{Building} data structure contains \texttt{building\_num},
\texttt{executed\_time} \& \texttt{total\_time}. \texttt{executed\_time}
denotes for how long the building was worked on. \texttt{twin} maintains
a pointer reference to corresponding node in red black tree.

\hypertarget{min-heap-interface}{%
\subsubsection{Min Heap Interface}\label{min-heap-interface}}

\begin{Shaded}
\begin{Highlighting}[]
\NormalTok{Building* insert(}\DataTypeTok{int}\NormalTok{ building_num, }\DataTypeTok{int}\NormalTok{ executed_time, }\DataTypeTok{int}\NormalTok{ total_time, Node* rbt_twin);}
\end{Highlighting}
\end{Shaded}

Creates a new nodes inside the min heap using \texttt{building\_num},
\texttt{executed\_time}, \texttt{total\_time}. \texttt{rbt\_twin} is the
reference to node in red black tree.

\begin{Shaded}
\begin{Highlighting}[]
\NormalTok{Building* removeMin();}
\end{Highlighting}
\end{Shaded}

Removes the minimum element from the min heap.

\begin{Shaded}
\begin{Highlighting}[]
\DataTypeTok{void}\NormalTok{ swap(Building *, Building *);}
\end{Highlighting}
\end{Shaded}

Swaps two building nodes, replacing their content with each other.

\begin{Shaded}
\begin{Highlighting}[]
\DataTypeTok{void}\NormalTok{ heapify();}
\end{Highlighting}
\end{Shaded}

Used or maintain order of min heap. If there's any violation wrt to min
heap, building node with minimum value is taken as root.

\begin{Shaded}
\begin{Highlighting}[]
\DataTypeTok{void}\NormalTok{ updateMin(}\DataTypeTok{int}\NormalTok{ time);}
\end{Highlighting}
\end{Shaded}

Increment the minimum node's \texttt{executed\_time} with the give
amount of \texttt{time}.

\begin{Shaded}
\begin{Highlighting}[]
\DataTypeTok{int}\NormalTok{ left_child(}\DataTypeTok{int}\NormalTok{ i);}
\end{Highlighting}
\end{Shaded}

Returns left child index.

\begin{Shaded}
\begin{Highlighting}[]
\DataTypeTok{int}\NormalTok{ right_child(}\DataTypeTok{int}\NormalTok{ i);}
\end{Highlighting}
\end{Shaded}

Returns left child index.

\hypertarget{red-black-tree}{%
\subsection{Red Black Tree}\label{red-black-tree}}

\hypertarget{red-black-tree-node-structure}{%
\subsubsection{Red black Tree Node
Structure}\label{red-black-tree-node-structure}}

\begin{Shaded}
\begin{Highlighting}[]
\KeywordTok{struct}\NormalTok{ Node \{}
    \DataTypeTok{int}\NormalTok{ building_num;}
    \DataTypeTok{bool}\NormalTok{ color;}
\NormalTok{    Node *left;}
\NormalTok{    Node *right;}
\NormalTok{    Node *parent;}
\NormalTok{    Building *twin;}

\NormalTok{    Node(}\DataTypeTok{int}\NormalTok{ building_num) \{}
        \KeywordTok{this}\NormalTok{->building_num = building_num;}
\NormalTok{        parent = left = right = NULL;}
\NormalTok{    \}}
\NormalTok{\};}
\end{Highlighting}
\end{Shaded}

Each instance of \texttt{Node} is data in red black. \texttt{Node}
consists of \texttt{building\_num} which would be unique within the data
structure. \texttt{color} refers to the type of node i.e \texttt{RED} or
\texttt{BLACK}. \texttt{left} and \texttt{right} node refers to the left
and right child of the red black tree. \texttt{parent} pointer maintains
reference to it's parent node. \texttt{twin} is used to store pointer
reference to corresponding node in min heap.

\hypertarget{red-black-tree-interface}{%
\subsubsection{Red Black Tree
Interface}\label{red-black-tree-interface}}

\begin{Shaded}
\begin{Highlighting}[]
\NormalTok{Node *insert(}\AttributeTok{const} \DataTypeTok{int}\NormalTok{ &n);}
\end{Highlighting}
\end{Shaded}

Inserts a node with building num \texttt{n} to Red Black Tree. Time
Complexity: \texttt{O(log\ n)}

\begin{Shaded}
\begin{Highlighting}[]
\NormalTok{Node *search(}\DataTypeTok{int}\NormalTok{ building_num);}
\end{Highlighting}
\end{Shaded}

Looks up a node with building num \texttt{building\_num} to Red Black
Tree. Time Complexity: \texttt{O(log\ n)}

\begin{Shaded}
\begin{Highlighting}[]
\BuiltInTok{std::}\NormalTok{string join(}\AttributeTok{const} \BuiltInTok{std::}\NormalTok{vector<}\BuiltInTok{std::}\NormalTok{string> & arr, }\AttributeTok{const} \DataTypeTok{char}\NormalTok{ * delim);}
\end{Highlighting}
\end{Shaded}

Method to join an array \texttt{arr} with delimiter \texttt{delim}. E.g:
\texttt{join(\{"Hi",\ "Hello"\},\ \textquotesingle{},\textquotesingle{})}
returns \texttt{Hi,\ Hello}. Time Complexity: \texttt{O(n)}

\begin{Shaded}
\begin{Highlighting}[]
\DataTypeTok{void}\NormalTok{ inorder(}\DataTypeTok{int}\NormalTok{ low, }\DataTypeTok{int}\NormalTok{ high);}
\end{Highlighting}
\end{Shaded}

Runs an inorder traversal on tree and looks up node with building number
between \texttt{low} and \texttt{high}. Time complexity:
\texttt{O(log\ n\ +\ (high\ -\ low))}

\begin{Shaded}
\begin{Highlighting}[]
\DataTypeTok{void}\NormalTok{ delete_node(Node *node);}
\end{Highlighting}
\end{Shaded}

Deletes the given node from red black tree. Time Complexity:
\texttt{O(log\ n)}

\begin{Shaded}
\begin{Highlighting}[]
\DataTypeTok{void}\NormalTok{ fix(Node *& node);}
\end{Highlighting}
\end{Shaded}

Fix violation in red black tree due to \texttt{node} on insert or
delete. Time Complexity: \texttt{O(log\ n)}

\begin{Shaded}
\begin{Highlighting}[]
\DataTypeTok{void}\NormalTok{ right_rotate(Node *);}
\end{Highlighting}
\end{Shaded}

Performs right rotation in red black tree. Time Complexity:
\texttt{O(1)}

\begin{Shaded}
\begin{Highlighting}[]
\DataTypeTok{void}\NormalTok{ left_rotate(Node *);}
\end{Highlighting}
\end{Shaded}

Performs left rotation in red black tree. Time Complexity: \texttt{O(1)}

\end{document}
